\documentclass[letterpaper, 10pt, conference]{ieeeconf}

\IEEEoverridecomanndlockouts

\overrrideIEEEmargins

\usepackage{authblk}
\usepackage{hyperref}

\title{\LARGE \bf
Working Title
}
\author[1]{Jose Navar}
\author[2]{Brendon Burnett}
\author[3]{Kenneth Romero}
\affil[1,2,3]{\emph{University of North Georgia}}

\begin{document}
    \maketitle
    \begin{abstract}
    In this paper, we prompt ChatGPT in different ways to assess if the responses from ChatGPT are reliable, factual, and
    unbiased. This would be extrapolated to see if ChatGPT could be reliably used in an academic setting such as a history lecture
    without producing false or biased information. 
    \end{abstract}

    \section{Introduction}
    The current education system often values memorization over conceptual understanding, rewarding those who can cram rather than those with a deep, long-term grasp of a subject. AI tools like GPT-4 can help educators interconnect subjects by quickly summarizing textbooks and open resources, facilitating a more personalized and streamlined approach to teaching. This can be particularly beneficial for students with disabilities and allows for more in-depth conversations with students as they use AI tools to identify their challenges. Inspired by a video by Fireship [1] our goal is to enable educators to use GPT-4 to develop modular and flexible education plans tailored to the specific needs of their students. While this approach can cater to most students, some groups may still require more personalized human interaction. As the printing press, libraries, and the internet/search-engines better proliferated the openness of academia, AI tools will only continue to do the same for us.

    \subsection{What is AI?}
    
    \bibliography{references}
\end{document}